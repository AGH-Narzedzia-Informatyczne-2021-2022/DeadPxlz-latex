\documentclass[../DeadPxlz.tex]{subfiles}
\graphicspath{{\subfix{../pictures/}}}
\usepackage{subfiles}

\begin{document}

\label{sec:tomasz}
\section{Rozdział Tomasza}

\begin{figure}[bh]
    \centering
    \includegraphics[width=7cm]{pictures/papuga.jpeg}
    \label{fig:papuga}
    \caption{Papuga}
\end{figure}

\textbf {Papugowe (Psittaciformes)} – rząd ptaków z podgromady Neornithes. Najliczniejsze w Ameryce Południowej i Australii. \underline{Zdjęcie~\ref{fig:papuga}to kolorowa papuga}

It is a long established fact that a reader will be distracted by the readable content of a page when looking at its layout. The point of using Lorem Ipsum is that it has a more-or-less normal distribution of letters, as opposed to using \underline{\textbf{'Content \emph{here}, content here',}} making it look like readable English. Many desktop publishing packages and web page editors now use Lorem Ipsum as their default model text, and a search for \emph{'lorem ipsum'} will uncover many web sites still in their infancy. Various versions have evolved over the years, sometimes by accident, sometimes on purpose (injected humour and the like).

\begin{itemize}
    \item Papuga występuje na kilku kontynentach na świecie - Australia, Afryka, Ameryka Południowa, Europa, Afryka
  \item Papuga jest roślinożercą podobnie jak zając, królik i chomik.
  \item Papuga to bardzo ciekawe zwierzę - potrafi imitować ludzką mowę.
\end{itemize}
\par

Wielkość momentu bezwładności wykorzystujemy w wielu rachunkach, np.:

\begin{itemize}
    \item[+] obliczanie środka masy ciała
  \item[+] obliczanie środka masy układu ciał
  \item[+] wyznaczanie osi obrotu ciała
  \item[+] tworzenie dźwigów i innych maszyn prostych
  \item[+] budowa mostów i innych budowli
\end{itemize}
    
\begin{Large}
    \centering
    \begin{equation}
        I = \sum_{i=1}^{n} m_{i}r{i}^2
        \label{bezwładność}
    \end{equation}
\end{Large}
~\ref{bezwładność} \centering \textit{Wzór na moment bezwładności papugi, gdzie}

\begin{enumerate}
    \item \( \sum \) to wzór sumy
    \item \( I \) to moment bezwładności 
    \item \( m \) to masa punktu materialnego
    \item \( r \) to odległość punktu materialnego od środka masy ciała
\end{enumerate}

\subfile{../tables/tabelatomasz}

\end{document}