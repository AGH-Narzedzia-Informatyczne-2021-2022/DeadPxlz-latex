\documentclass[../DeadPxlz.tex]{subfiles}
\graphicspath{{\subfix{../pictures/}}}

\begin{document}

\label{sec:piotr}
\section{Rozdział Piotra}

\begin{figure}[!ht]
    \centering
    \includegraphics[width=10cm]{octopus_ch_piotr.jpg}
    \label{fig:octopus}
    \caption{Ośmiornica zwyczajna}
\end{figure}

\vspace{5pt}

\begin{Large}
    \begin{equation}
        \label{eq:sequence}
        a_n=\prod_{k=1}^{n}{\left(1+\frac{1}{2^k}\right)}, n \in \mathbb{N}
    \end{equation}
\end{Large}

\begin{Large}
    \begin{equation}
        \label{eq:property}
        \forall_{n \in \mathbb{N}} \: a_n=\prod_{k=1}^{n}{\left(1+\frac{1}{2^k}\right)} < e
    \end{equation}
\end{Large}

\subfile{../tables/tabelapiotr}

\begin{figure}
    \label{fig:info}
    \centering
    Podstawowe informacje na temat ciągu $a_n$:\\
    \begin{itemize}
        \item ciąg $a_n$ jest rosnący,
        \item ciąg $a_n$ jest ograniczony,
        \item ciąg $a_n$ ma granicę.
    \end{itemize}
    \caption{Informacje o ($a_n$)}
\end{figure}

\newpage
\begin{figure}[!ht]
    \label{fig:proof}
    \centering
    Aby udowodnić te własności wystarczy:\\
    \begin{enumerate}
        \item pokazać, że $\forall_{n \in \mathbb{N}} a_n>0$,
        \item pokazać, że $\forall_{n \in \mathbb{N}} \frac{a_{n+1}}{a_n}>1$,
        \item pokazać, że ciąg $\left( a_n \right)$ jest ograniczony, korzystając z nierówności między średnimi,
        \item z dwóch poprzednich własności wynika, że ciąg $\left( a_n \right)$ ma granicę.
    \end{enumerate}
    \caption{Dowód własności ($a_n$)}
\end{figure}

\setlength{\parindent}{5ex}
\justifying
\par
W celu zachowania spójności tekstu, wstawione zostało zdjęcie \textit{ośmiornicy zwyczajnej} (\ref{fig:octopus}), będącej \emph{zwierzęciem}. Treść rozdziału nie będzie jednak związana z tymi mięczakami.
\par
Tematem tego rozdziału jest podanie podstawowych informacji na temat \textbf{ciągu} \ref{eq:sequence}. Podana została \underline{własność \ref{eq:property}} oraz $3$ podstawowe informacje na jego temat (\ref{fig:info}). Kilka początkowych wyrazów ciągu zostało umieszczonych w tabeli \ref{tab:sequence}. Podany jest również schemat dowodzenia opisanych własności (\ref{fig:proof}).

\end{document}