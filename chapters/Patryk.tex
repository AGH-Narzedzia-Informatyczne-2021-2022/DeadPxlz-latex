\documentclass[../DeadPxlz.tex]{subfiles}
\graphicspath{{\subfix{../pictures/}}}
\usepackage{subfiles}
\usepackage{amsmath}
\usepackage{indentfirst}

\begin{document}

\label{sec:patknap}
\section{Rozdział Patryka}

\begin{figure}[h!]
    \centering
    \includegraphics[width=7cm]{pictures/catt.jpg}
    \label{fig:Katt}
    \tiny \caption{Koty są to majestatyczne stworzenia}
\end{figure}

\large
\textbf{Katt} (Felis catus), även känd som tamkatt, är ett relativt litet, smygjagande rovdjur i familjen kattdjur och ett vanligt sällskapsdjur i stora delar av världen. Falbkatten (Felis silvestris lybica), en underart till vildkatten (Felis silvestris), blev tidigt domesticerad av människan som nyttodjur för att hålla efter skadedjur eller för sällskap, både på landet och i stadsmiljö. 

Table~\ref{tab:tabelapat} zawiera losowe liczby.
\subfile{../tables/tabelapat}

\textit{Till havs har man sedan gammalt haft skeppskatter som skyddade mot gnagarangrepp på tåg och i proviantförråd. Katter kan dessutom vara underhållande för många människor med sin tillgivenhet och används även i terapeutiskt syfte vid psykisk ohälsa.
(From Wikipedia, about house cat, in Swedish)}

The length of a cat's jump can be computed with a formula: \\
\centering
    \begin{math}
        \frac{F_k*T_r^2}{M_k}
    \end{math}
\clearpage

Popularne rasy kotów:
\begin{itemize}
  \item Brytyjski
  \item Perski
  \item Sfinks
\end{itemize}

Co trzeba zapewnić kotu:
\begin{enumerate}
  \item Odpowiednio zbilansowana dieta
  \item Opieka weterynarza
  \item Ciepły kącik
\end{enumerate}


\end{document}