\documentclass[../DeadPxlz.tex]{subfiles}
\graphicspath{{\subfix{../pictures/}}}
\usepackage{subfiles}

\begin{document}

\begin{figure}[bh]
    \centering
    \includegraphics[width=7cm]{ropucha}
    \label{fig:img1}
    \caption{ropucha}
\end{figure}

\textbf{Ropucha szara, ropucha zwyczajna (Bufo bufo)} – gatunek płaza, największa z żyjących w Polsce ropuch. \underline{Zdjęcie ~\ref{fig:img1} to ropucha szara}

U tego gatunku dymorfizm płciowy pod względem wielkości ciała jest najwyraźniej zaznaczony wśród polskich ropuch. Samce – mniejsze – osiągają długość 46-98 mm, samice – większe – długość 61-125 mm. Ciało tego płaza jest krępe i masywne, pysk szeroki. Skóra grzbietu chropowata ze względu na liczne brodawki. Z tyłu głowy olbrzymie gruczoły przyuszne. Błony bębenkowe są małe i słabo widoczne. Na środkowych palcach tylnych nóg występują podwójne modzele stawowe. Błony pławne sięgają do połowy palców. Zabarwienie grzbietu jest brązowe w różnych odcieniach szarości, zwykle jednolite. Brzuch zawsze jaśniejszy, brudnoszary, pokryty plamami.

\begin{itemize}
  \item Ropucha szara jest płazem z rodziny ropuchowatych.
Do rodziny tej należy ponad 350 gatunków podzielonych na 26 rodzajów.
  \item Jest rozpowszechniona w całej Europie kontynentalnej i Azji środkowej. Występują również populacje północnoafrykańskie skupione wokół basenu Morza Śródziemnego u wybrzeży Maroko i Algierii.
  \item Ropucha szara może osiągać do 15 cm długości ciała. Populacje północne osiągają rozmiary mniejsze od południowych.
Samice są przeważnie grubsze i większe od samców.
\end{itemize}
\par

\begin{Large}
    \begin{equation}
        \sum_{k=0}^{n-1} e^{\frac{2ik\pi}{n}} = 0
        \label{euler}
    \end{equation}
\end{Large}
~\ref{euler} \textit{Tożsamość Eulera}

gdzie
\begin{enumerate}
    \item \( e \) to liczba Eulera , podstawa logarytmów naturalnych ,
    \item \( i \) jest jednostką urojoną , która z definicji spełnia \( i^2 = -1 \) , oraz
    \item \( \pi \) jest PI The stosunek z obwodu z koła do jego średnicy .
\end{enumerate}
Tożsamość Eulera jest często przytaczana jako przykład głębokiego matematycznego piękna. Trzy podstawowe operacje arytmetyczne występują dokładnie raz: dodawanie, mnożenie i potęgowanie.\par


\subfile{../tables/tabelak}

Tabela ~\ref{tabela} jest tabelą prawdy

Tablicę prawdy posiadają wszystkie funktory zdaniotwórcze, np.: 

\begin{itemize}
  \item[*]  alternatywa (OR)
  \item[*] alternatywa wykluczająca (XOR)
  \item[*] implikacja logiczna
  \item[*]  koniunkcja logiczna (AND)
  \item[*] negacja (NOT)
  \item[*] równoważność (XNOR)
  \item[*] dysjunkcja (NAND)
  \item[*] binegacja (NOR).


\end{itemize}



\end{document}