\documentclass[../DeadPxlz.tex]{subfiles}
\graphicspath{{\subfix{../pictures/}}}
\usepackage{subfiles}

\begin {document}
\section{Rozdział Wojciecha}

\begin{figure}[h]
\centering
\includegraphics[width=8cm]{pictures/guziec.jpg}
\caption{Guziec zwyczajny}
\end{figure}

\textbf{Guziec} (Phacochoerus) – rodzaj ssaka z podrodziny świń (Suinae) w obrębie \underline{rodziny świniowatych (Suidae).} 
\par Długość ciała samic 105–140 cm, samców 125–150 cm, długość ogona 35–50 cm, długość górnych kłów u samców 25– 30 cm (rekordowo 60 cm), wysokość w kłębie 55–85 cm; masa ciała samic 50–75 kg, samców 60–150 kg. 

\vspace{1.5cm}

Do rodzaju należą następujące gatunki:
\begin {enumerate}
\item Phacochoerus africanus (J.F. Gmelin, 1788) – guziec zwyczajny
\item Phacochoerus aethiopicus (Pallas, 1766) – guziec pustynny
\end{enumerate}

\newpage
\textbf{Etymologia}
\begin {itemize}
    \item[->]Aper
    \item[->]Macrocephalus
    \item[->]Eureodon
    \item[->]Phacochoerus (Phacochaeres, Phacochaeres, Phascochoeres)
    \item[->]Dinochoerus
    \item[->]Phacellochaerus (Phacellochoerus)
\end{itemize}

\vspace {1.5cm}
\begin{Large}
    \centering 
    \begin{equation}
    F_{g} = G\frac{m_{1}m_{2}}{R^2}
    \label{wzor}
    \end{equation}
\end{Large}
~\ref{wzor} \centering {Wartość siły grawitacji}
\vspace {1cm}
\subfile{../tables/tabelaWN}

\end {document}